\chapter{پیاده‌سازی و توسعه مدل یادگیری ماشین}

در این فصل روش پیاده‌سازی مدل هوش مصنوعی را به تفصیل شرح خواهیم داد. ابتدا فرضیات و داده‌های ورودی و آماده‌ی تحلیل را مشخص کرده و نماد هر کدام را که تا انتهای این نوشته از آنها استفاده خواهیم کرد، مشخص می‌کنیم. در قسمت بعد مراحل پیش‌پردازش\LTRfootnote{Preprocess} را که روی این داده‌ها انجام می‌شود به ترتیب توضیح می‌دهیم و سپس ویژگی‌هایی که نیاز داریم از این داده‌های خام دربیاوریم را توضیح می‌دهیم و نحوه‌ی استخراج این ویژگی‌ها را نمایان می‌کنیم. در مرحله‌ی نهایی روش آموزاندن و یادگیری مدل هوش مصنوعی را بر اساس این ویژگی‌ها شرح می‌دهیم.


\section{توضیح مسئله}
با توجه به اینکه سیستم یادگیری ماشین بر اساس اطلاعات حسگرهای لرزش عمل می‌کند، برای داشتن کمترین خطا در عملیات پیش‌بینی باید فرضیاتی را پیش از طراحی و پیاده‌سازی سیستم در نظر داشته باشیم. اولاً نمونه‌های بدست‌آمده برای حسگرهای متفاوت بازه‌های زمانی مختلف را در بر می‌گیرند و همگن نیستند. ثانیاً این داده‌های دارای انحرافاتی در اندازه‌گیری بدلیل وجود گرانش یا خرابی حسگر هستند. ثالثاً وضعیت ابتدایی هر یک از گره‌هایی که می‌خواهیم اطلاعات لرزش آنها را جمع‌آوری و تحلیل کنیم یکی نیستند\cite{jung2017vibration}. با توجه به نکاتی که مطرح کردیم، پیاده‌کردن یک سیستم پیش‌پردازش و استخراج‌کننده‌ی ویژگی‌های مناسب، الزامی است. در \cref{table:notation_description} توضیحات نشانه‌گذاری داده‌ی مربوط به این مسئله را می‌بینیم.

\begin{table}[h!]
  \begin{center}
    \caption{توضیحات نشانه‌گذاری داده‌ها}
    \label{table:notation_description}
    \begin{tabular}{|c|c|} % <-- Alignments: 1st column left and 2nd right with vertical lines in between
    	\hline
تعداد کل گره‌ها & $N$\\
    	\hline
 تعداد کل اندازه‌گیری‌ها & $M$\\
    	\hline
   تعداد کل نمونه‌های یک اندازه‌گیری & $K$\\
    	\hline
گره $n$ ام & $n$\\
 	\hline
اندازه‌گیری $m$ ام & $m$\\
 	\hline
نمونه‌ی $k$ ام یک اندازه‌گیری & $k$\\
 	\hline
بردار سه‌بعدی مربوط به اندازه‌گیری لرزش & $a_{nmk}$\\
 	\hline
بردار $k$بعدی مربوط به لرزش در محور $l \in \{x, y, z\}$  & $a^l_{nm}$\\
 	\hline
    \end{tabular}
  \end{center}
\end{table} 

\section{پیش‌پردازش}
این بخش وظیفه دارد قبل از انجام تحلیل داده، انحرافات و داده‌های پرت\LTRfootnote{Outlier Data} را از داده‌ی خام جدا کرده و داده‌ی قابل پردازش را به لایه‌ی بعد که لایه‌ی استخراج ویژگی‌ است تحویل دهد.

\subsection{از بین بردن انحرافات}
 حسگرهای کم‌هزینه \lr{MEMS} ،که داده‌های جمع‌آوری‌شده برای این پروژه توسط این نوع از حسگرها تأمین شده است، غالباً با گذشت زمان دچار انحرافاتی در اندازه‌گیری خواهند شد که منجر به اضافه یا کم شدن یک مقدار شتاب غیر صفر در اندازه‌گیری‌هایشان خواهد شد. از طرفی وجود گرانش، تاثیراتی روی اندازه‌گیری‌ها خواهد داشت و موجب ایجاد انحرافاتی رو به بالا یا پایین در این مقادیر خواهد شد\cite{jung2017vibration}. برای از بین بردن این مشکل همانطور که در \cref{eq:normalize} آورده شده است، از عادی‌کردن\LTRfootnote{Normalizing} داده با کم‌کردن میانگین مقادیر شتاب اندازه‌گیری شده در هر کدام از سه محور از مقادیر اندازه‌گیری‌شده استفاده کرد. لازم به ذکر است همانطور که مشخص است، $\hat{a}^l_{nm}$ نماد ماتریس عادی‌شده است.
\begin{equation}
\label{eq:normalize}
	\hat{a}^l_{nm}=a^l_{nm}-\sum_{k=1}^K \dfrac{a^l_{nmk}}{K}
\end{equation}


\subsection{از بین بردن داده‌های پرت}


\section{استخراج ویژگی‌ها}


\section{نحوه‌ی یادگیری مدل}