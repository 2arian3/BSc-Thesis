%% -!TEX root = AUTthesis.tex
% در این فایل، عنوان پایان‌نامه، مشخصات خود، متن تقدیمی‌، ستایش، سپاس‌گزاری و چکیده پایان‌نامه را به فارسی، وارد کنید.
% توجه داشته باشید که جدول حاوی مشخصات پروژه/پایان‌نامه/رساله و همچنین، مشخصات داخل آن، به طور خودکار، درج می‌شود.
%%%%%%%%%%%%%%%%%%%%%%%%%%%%%%%%%%%%
% دانشکده، آموزشکده و یا پژوهشکده  خود را وارد کنید
\faculty{دانشکده مهندسی کامپیوتر}
% گرایش و گروه آموزشی خود را وارد کنید
\department{گرایش معماری سیستم‌های کامپیوتری}
% عنوان پایان‌نامه را وارد کنید
\fatitle{پیاده‌سازی سیستم نگهداری و تعمیرات پیش‌بینانه تجهیزات بر بستر اینترنت اشیاء مبتنی بر تحلیل لرزش}
% نام استاد(ان) راهنما را وارد کنید
\firstsupervisor{دکتر حمیدرضا زرندی}
%\secondsupervisor{استاد راهنمای دوم}
% نام استاد(دان) مشاور را وارد کنید. چنانچه استاد مشاور ندارید، دستور پایین را غیرفعال کنید.
%\firstadvisor{حامد فربه}
%\secondadvisor{استاد مشاور دوم}
% نام نویسنده را وارد کنید
\name{میلاد }
% نام خانوادگی نویسنده را وارد کنید
\surname{اسرافیلیان نجف‌آبادی}
%%%%%%%%%%%%%%%%%%%%%%%%%%%%%%%%%%
\thesisdate{تیر ۱۴۰۲}

% چکیده پایان‌نامه را وارد کنید
\fa-abstract{
موفقیت اینترنت اشیاء به توانایی ما در حل مشکلات چالش برانگیز که قبلا غیرممکن بودند بستگی دارد. یکی از حیاتی‌ترین چالش ها در فضای اینترنت اشیاء، نگهداری پیش‌گیرانه در فرآیندهای تولید صنعتی برای به حداکثر رساندن در دسترس بودن و دوام تجهیزات است. به طور سنتی، مدیریت پیش‌گیرانه فقط از استراتژی‌های کم‌هزینه‌تر، مثلاً مدیریت مبتنی بر زمان یا استفاده، پیروی می‌کند. با حجم زیادی از داده‌های حسگر جمع‌آوری‌شده از اینترنت اشیا، می‌توانیم تعمیر و نگهداری پیش‌بینی‌کننده بسیار هوشمندتری را بر اساس پیش‌بینی دقیق خرابی ماشین‌آلات توسعه دهیم. در این پروژه، ما سیستمی را متشکل از یک کارگزار اصلی و یک بسته‌ی یادگیری ماشین بر حسب داده‌های حسگرهای لرزش برای پیاده‌سازی مدیریت پیش‌بینانه توسعه داده‌ایم. به طور دقیق‌تر فرآیند ساخت مدل یادگیری ماشین را بر اساس پیش‌پردازش داده‌های حسگرها، استخراج ویژگی‌های مناسب و سپس تحلیل این ویژگی‌ها پیش‌ بردیم. در نهایت با یادگیری مدلی خطی بر اساس تحلیل و مقایسه‌ی این ویژگی‌ها با دستگاه‌های سالم، زمان مفید باقی‌مانده برای هر گره موجود را پیش‌بینی کردیم. با استفاده از چنین سیستمی در صنعت، هزینه‌های نگهداری و همچنین ضررهای احتمالی متحمل‌شده بشدت کاهش خواهد یافت.
}


% کلمات کلیدی پایان‌نامه را وارد کنید
\keywords{نگهداری پیش‌بینانه، تحلیل لرزش، یادگیری ماشین، اینترنت اشیاء}



\AUTtitle
%%%%%%%%%%%%%%%%%%%%%%%%%%%%%%%%%%
\vspace*{7cm}
\thispagestyle{empty}
\begin{center}
\includegraphics[height=5cm,width=12cm]{besm}
\end{center}