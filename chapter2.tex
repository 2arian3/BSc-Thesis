\chapter{مؤلفه‌های سیستم و تکنولوژی‌ها}

\section{مؤلفه‌های سیستم طراحی‌شده}
در این بخش ابتدا به طریقه‌ی استقرار\LTRfootnote{Deploy} مدل یادگیری ماشین سیستم پیاده‌شده اشاره می‌کنیم و در قسمت‌های بعد اجزای اصلی این دستگاه را به طور دقیق مورد بررسی قرار می‌دهیم. شایان ذکر است که مدل یادگیری ماشین به صورت یک سرویس مجزا به کارگزار\LTRfootnote{Server} جمع‌آوری اطلاعات لرزش گره‌ها که در یک پروژه‌ی مجزا در راستای پروژه‌ی کنونی توسعه یافته است، اضافه شده است.

\subsection{نحوه‌ی استقرار مدل یادگیری ماشین}
 روش‌های مختلفی برای استقرار و استفاده از مدل‌های یادگیری هوش مصنوعی مورد استفاده قرار می‌گیرند که از بین اینها چهار روش نشان داده‌ شده در \cref{fig:ml_model_deployments} مرسوم‌تر هستند\cite{kaggleMLdeployments}:

\begin{figure}[!h]
\centerline{\includegraphics[width=.5\textwidth]{ml_model_deployments.png}}
\caption{انواع روش‌های استقرار مدل‌های یادگیری ماشین}
\label{fig:ml_model_deployments}
\end{figure}

\begin{itemize}


\item \textbf{پیاده‌سازی دسته‌ای\LTRfootnote{Batch Deployment}}: پیش‌بینی‌ها به فاصله‌های زمانی مشخص محاسبه می‌شوند و پیش‌بینی‌های حاصل در پایگاه داده ذخیره می‌شوند و به راحتی می‌توان آنها را در صورت نیاز بازیابی کرد. با این حال، نمی‌توان از داده‌های بروزتر استفاده کرد و پیش‌بینی‌ها می‌توانند به سرعت منسوخ شوند\cite{singh2021deploy, pacheco2018towards}.

\item \textbf{پیاده‌سازی بی‌درنگ\LTRfootnote{Real-Time Deployment}}: در این نوع از استقرار، درخواست کاربر برای گرفتن جدید‌ترین پیش‌بینی‌ها به عنوان یک راه‌انداز\LTRfootnote{Trigger} توسط رابط برنامه‌نویسی\LTRfootnote{Application Programming Interface (API)} اچ‌تی‌تی‌پی\LTRfootnote{Hypertext Transfer Protocol (HTTP)} به کارگزار ارسال می‌شود. سپس سرویس یادگیری ماشین که به عنوان افزونه‌ای در سمت کارگزار توسعه یافته است، شروع به کار می‌کند و جدیدترین نتایج پیش‌بینی را تولید و ذخیره می‌کند و به سمت کاربر به عنوان نتیجه ارسال می‌کند. مشکل اصلی این روش قرارگیری مدل یادگیری ماشین، کند بودن روند یادگیری و پیش‌بینی است که منجر به منتظر ماندن کاربر می‌گردد. می‌توان با بهره‌گیری از فرآیند\LTRfootnote{Process}‌های چندریسمانی\LTRfootnote{Multi-Threaded} برای دریافت درخواست‌های کاربر و انجام مرحله‌ی یادگیری و پیش‌بینی مدل، تا حد زیادی این مشکل را برطرف کرد\cite{singh2021deploy, pacheco2018towards}.

\item \textbf{پیاده‌سازی جریانی\LTRfootnote{Streaming Deployment}}: این امکان را می‌دهد تا فرآیند ناهمزمان‌\LTRfootnote{Asynchronous}تری ایجاد شود. یک رویداد می‌تواند شروع فرآیند استنتاج را فراهم کند. این فرآیند در صف یک واسط پیام\LTRfootnote{Message Broker} مانند کافکا\LTRfootnote{Apache Kafka} قرار داده می‌شود و مدل یادگیری ماشینی در هنگام آماده شدن برای انجام درخواست، آن را انجام می‌دهد. این کار به سرویس پشتیبانی فرصت می‌دهد و با فرآیند صف بهینه، قدرت محاسباتی بسیاری را صرفه‌جویی می‌کند. پیش‌بینی‌های حاصل شده نیز در صف قرار گرفته و در صورت نیاز توسط سرویس‌های پشتیبانی مصرف می‌شوند. از مزیت‌های این روش نسبت به روش بی‌درنگ، می‌توان به کم‌شدن تاخیر پاسخ‌دهی به کاربران اشاره کرد\cite{singh2021deploy, pacheco2018towards}.

\item \textbf{پیاده‌سازی لبه‌ای\LTRfootnote{Edge Deployment}}: در این روش استقرار، مدل مستقیماً بر روی کلاینت نصب می‌شود، مانند مرورگر وب، یک تلفن همراه یا محصولات اینترنت اشیاء. این کار باعث رسیدن به سریع‌ترین استنتاج می‌شود، اما معمولاً مدل‌ها باید به اندازه کافی کوچک باشند تا بتوانند در سخت‌افزارهای کوچکتر نصب شوند\cite{kaggleMLdeployments}.

\end{itemize}

\subsection{بارگیری مراجع}
در ابتدا مراجع را باید از سایت‌های معتبر بارگیری کنیم، مثلا برای ارجاع دادن به مقاله‌ی
\lr{A classification of some Finsler connections and their applications}
ابتدا به سایت
\href{scholar.google.com}{گوگل اسکولار} 
رفته و این مقاله را جستجو می‌کنیم. پس از پیدا کردن این مقاله، مانند شکل زیر، در زیر نام و چکیده‌ی مقاله، $5$ گزینه وجود دارد که عبارتند از:\\

\begin{enumerate}
\item \lr{ Cited by}

\item \lr{ Related articles}

\item \lr{ All 6 versions}

\item \lr{ Cite}

\item \lr{ Save}
\end{enumerate}
\begin{figure}[!h]
\includegraphics[height=3cm]{bidabad}
\caption{نمونه یک مقاله در گوگل اسکولار}
\end{figure}
در اینجا ما به گزینه‌ی چهارم یعنی
\lr{ Cite}
احتیاج داریم. بر روی آن کلیک کرده و پنجره‌ای مانند
\cref{fig.2}
باز می‌شود که دارای $4$ گزینه‌ی زیر است:
\begin{enumerate}
\item \lr{BibTeX}

\item \lr{EndNote}

\item \lr{RefMan}

\item \lr{RefWorks}
\end{enumerate}
\begin{figure}
\centering\includegraphics[scale=.6]{bibref}
\caption{پنجره‌ی باز شده در گوگل اسکولار}\label{fig.2}
\end{figure}
روی گزینه‌ی اول، یعنی
\verb;BibTeX;
کلیک کرده و همه‌ی نوشته‌های پنجره‌ی باز شده را مانند زیر، کپی کرده و در فایل
\verb;references.bib;
موجود در فایل
\verb;AUTthesis;
پیست می‌کنیم. سپس کلیدهای
\verb;Ctrl+s;
را می‌زنیم تا فایل ذخیره شود.\\
\begin{latin}
	\normalsize
\begin{verbatim}
@ article{bidabad2007classification,
title={A classification of some Finsler connections and their applications},
author={Bidabad, Behroz and Tayebi, Akbar},
journal={arXiv preprint arXiv:0710.2816},
year={2007}
}
\end{verbatim}
\end{latin}
\subsection{روش ارجاع در متن}
برای ارجاع دادن به مقاله‌ی بالا، باید در جایی که می‌خواهید ارجاع دهید، دستور زیر را تایپ کنید:
\begin{latin}
\lr{$\backslash$cite\{bidabad2007classification\}}
\end{latin}
همانطور که مشاهده می‌کنید از کلمه‌ای که در سطر اول ادرس مقاله آمده (یعنی کلمه‌ی پس از
\lr{@article$\lbrace$})
استفاده کرده‌ایم. پس از دستور فوق، به صورت \cite{bidabad2007classification} و \cite{aa} مرجع خواهد خورد. توجه شود که در صورتی مراجع چاپ خواهند شد که در متن به انها ارجاع داده شده باشد. همچنین برای ارجاع چندتایی از دستور 
\lr{$\backslash$cite\{name1, name2,...\}}
استفاده کنید که به‌صورت \cite{najafi2008finsler, zakeri, najafi} ارجاع خواهند خورد.
\subsection{روش اجرای برنامه}
ابتدا فایل
\verb;AUT_thesis.tex;
را باز کرده و آن را دو بار اجرا کنید. سپس حالت اجرا را از 
\verb;Build Quick;
به حالت
\verb;Bibtex;
تغییر داده و دوباره برنامه را اجرا کنید. دو بار دیگر برنامه را در حالت 
\verb;Build Quick;
اجرا کرده و نتیجه را مشاهده کنید. در این روش تمامی مراجع بر اساس اینکه کدام یک در متن زودتر به آن ارجع داده شده لیست خواهند شد.
\subsection{مراجع فارسی}
برای نوشتن مراجع فارسی باید به صورت دستی، در همان فایل قبلی به صورت زیر عمل می‌کنیم:
\begin{LTR}
\noindent\verb;@article{manifold,;\\
\verb;title={;منیفلد هندسه\verb;},;\\
\verb;author={;بیدآباد دکتربهروز \verb;},;\\
\verb;journal{; امیرکبیر صنعتی دانشگاه\verb;},;\\
\verb;year={1389},;\\
\verb;LANGUAGE={Persian};\\
\verb;};
\end{LTR}
همانطور که مشاهده می‌کنید تنها تفاوت آن با حالت مراجع انگلیسی، سطر آخر آن می‌باشد که زبان را مشخص می‌کند که حتماً باید نوشته شود.
\section{راهنمای واژه‌نامه}

به دلیل پیچیدگی واژه‌نامه‌های موجود در سایت پارسی لاتک، از روش زیر برای نوشتن واژه‌نامه استفاده کنید:

ابتدا با استفاده از اکسل، واژه های خود را یک‌بار براساس حروف الفبای فرسی و بار دیگر انگلیسی مرتب کنید. سپس واژه ها را در فایل \lr{dicen2fa} و \lr{dicfa2en} قرار دهید.

\section{ساخت نمایه}\label{Namaye}
\subsection{ساخت نمایه}
 \begin{enumerate}

\item
کلمات مورد نظر خود مثلا \lr{word} با دستور \verb|\index{word}| ایندکس کنید.
\item
نحوه‌ی اجرای \lr{Make Index}   در ویرایشگرهای \lr{TeX Maker} و \lr{TeX Works}:
\begin{itemize}
\item  تک‌میکر: از منوی \lr{Tools} گزینه‌ی \lr{Xindy Make Index} را کلیک کنید یا از دکمه‌‌های میانبر \lr{Ctrl+Alt+I} استفاده کنید.

\item  تک‌ورکز: ابتدا باید مثل عکس زیر تنظیم  و سپس گزینه‌ی \lr{Xindy Make Index}  انتخاب و روی دکمه‌ی سبز رنگ کلیک کنید یا از دکمه‌های  \lr{Ctrl+T} استفاده کنید.

\begin{figure}[!h]
\centerline{\includegraphics[width=.5\textwidth]{Xindy_Make_Index.png}}
\caption{تنظیمات مربوط به تک‌ورکز}
\end{figure}

\end{itemize}
 \end{enumerate}
 
 \index{کتاب}
\index{پارسی‌لاتک}
\index{بی‌دی}
\index{سوال}
\index{عنصر}
\index{گزینه}
\index{ژاکت}
\index{مرکز دانلود}
\index{اجرا}
\index{تک‌لایو}
\index{ثالث}
\index{جهان}
\index{چهار}
\index{حمایت}
\index{خواهش}
\index{دنیا}
\index{زی‌پرشین}
\index{ریحان}
\index{شیرین}
\index{صمیمی}
\index{ضمیر}
\index{طبیب}