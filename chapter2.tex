\chapter{تکنولوژی‌های استفاده‌شده}

در این فصل تکنولوژی‌ها و چارچوب\LTRfootnote{Framework}‌های اصلی دخیل در توسعه این دستگاه را به طور دقیق مورد بررسی قرار می‌دهیم.

\section{‌‌زبان برنامه‌نویسی}
برای انتخاب زبان برنامه‌نویسی مناسب برای توسعه مدل یادگیری ماشین شرح‌داده شده، باید معیارهای متفاوتی را در نظر گرفت. برای این منظور زبان پایتون\LTRfootnote{\href{https://docs.python.org/3/}{Python}} را برگزیدیم. مواردی همچون داشتن چارچوب‌ها و کتابخانه‌های قدرتمند یادگیری ماشین، توسعه‌ی آسان و سریع و محبوبیت بالا از دلایل اصلی انتخاب پایتون به عنوان زبان اصلی برای توسعه‌ی سرویس یادگیری ماشین می‌باشد. همچنین شایان ذکر است که چون کارگزار اصلی جمع‌آوری اطلاعات لرزش به زبان پایتون نوشته شده است، استفاده از این زبان برای توسعه مدل یادگیری ماشین، باعث بهبود توسعه‌پذیری نیز می‌گردد. 

\subsection{زبان برنامه‌نویسی پایتون}
یک زبان برنامه‌نویسی عمومی و سطح بالا است که فلسفه طراحی آن بر روی خوانایی کد تأکید دارد. نحو\LTRfootnote{Syntax} پایتون به برنامه‌نویسان امکان می‌دهد تا مفاهیم را با تعداد کمتری خط کد نسبت به زبان‌هایی مانند سی\LTRfootnote{C Programming Language} بیان کنند و این زبان ساختارهایی را فراهم می‌کند که برنامه‌های واضح و قابل فهم را در هر دو مقیاس کوچک و بزرگ فراهم می‌سازد\cite{van2007python}. یکی از مشخصه‌های مهم پایتون این است که از چندین الگو\LTRfootnote{Paradigm}ی برنامه‌نویسی، از جمله شیءگرا\LTRfootnote{Object Oriented Programming (OOP)} و تابعی یا روش‌های رویه‌ای، پشتیبانی می‌کند. پایتون سیستم نوع پویا و مدیریت خودکار حافظه را پشتیبانی می‌کند و کتابخانه‌های استاندارد و جانبی بزرگ و جامع دارد. مفسرهای پایتون برای بسیاری از سیستم‌عامل‌ها در دسترس هستند\cite{srinath2017python}. از جمله مهم‌ترین ویژگی‌های پایتون می‌توان به موارد زیر اشاره کرد.

\begin{itemize}

\item \textbf{سادگی}: پایتون یک زبان برنامه‌نویسی بسیار سطح بالا است که منابع زیادی برای یادگیری آن وجود دارد. پایتون از ابزارهای شخص ثالث متنوعی پشتیبانی می‌کند که استفاده از آن را بسیار آسانتر می‌کند و کاربران را ترغیب می‌کند تا ادامه دهند\cite{srinath2017python, sharma2020python}.

\item \textbf{متن‌باز بودن\LTRfootnote{Open Source}}: اگرچه تمام حقوق این زبان برنامه‌نویسی متعلق به سازمان پایتون است، اما درحال‌ حاضر به عنوان یک نرم‌افزار متن‌باز وجود دارد و هیچ محدودیتی در استفاده، تغییر و توزیع آن وجود ندارد. می‌توان به آزادی از پایتون استفاده کرد و آن را برای استفاده شخصی و یا تجاری توزیع کرد. نه تنها می‌توان نرم‌افزاری که با آن نوشته شده است را استفاده و توزیع کرد، بلکه حتی می‌توان تغییراتی در خود کد منبع پایتون اعمال کرد. همچنین شایان ذکر است که پایتون یک جامعه بزرگ و پویا دارد که در هر نسخه آن را بهبود می‌بخشد\cite{srinath2017python, sharma2020python}.

\item \textbf{کتابخانه‌ها و چارچوب‌ها}: پایتون دارای یک سری کتابخانه‌های استاندارد و چارچوب‌های متنوع است که کار برنامه‌نویسان را بشدت راحت می‌کند، زیرا نیازی نیست تمام کدنویسی را خود برنامه‌نویس انجام دهد. کتابخانه‌های استاندارد در پایتون به خوبی تست شده‌اند و توسط هزاران نفر استفاده می‌شوند. بنابراین، می‌توان اطمینان داشت که استفاده از این کتابخانه‌ها توانایی ایجاد خرابی در برنامه‌های شما را ندارند\cite{srinath2017python, sharma2020python}.

\end{itemize}

حال به بررسی معایب پایتون می‌پردازیم. نکته‌ی قابل توجه در این قسمت این است که اگر معایب نام‌برده شده تاثیر زیادی در کیفیت خدمت ارائه‌شده به کاربر بگذارند، استفاده از پایتون اصلا توصیه نمی‌شود و باید به دنبال جایگزینی مناسب گشت. از جمله کاستی‌های پایتون عبارت‌اند از:

\begin{itemize}

\item \textbf{کندی}: به عنوان یک زبان با نوع پویا، پایتون به دلیل انعطاف‌پذیری بالا، کند عمل می‌کند، زیرا ماشین باید بسیاری از مراجعات را انجام دهد تا از تعریف چیزی مطمئن شود و این باعث کاهش عملکرد پایتون می‌شود\cite{srinath2017python, sharma2020python}.

\item \textbf{دشواری فرایند نگهداری\LTRfootnote{Maintaining}}: به دلیل اینکه پایتون یک زبان با نوع پویا است، یک چیز ممکن است به راحتی به معنای متفاوتی در تک‌نمایی متفاوت تفسیر شود. با افزایش اندازه و پیچیدگی یک برنامه پایتون، نگهداری آن ممکن است دشوار شود. با کمک تست‌های واحد\LTRfootnote{Unit Tests} می‌توان تا حدی این از وقوع این مشکل جلوگیری کرد\cite{srinath2017python, sharma2020python}.

\end{itemize}



\section{چارچوب‌ها و کتاب‌خانه‌ها}
در این پروژه از چارچوب فست‌ای‌پی‌آی\LTRfootnote{FastAPI} برای دریافت درخواست‌ها و ارسال نتایج پیش‌بینی استفاده‌شده است. این سرویس به عنوان یک بسته\LTRfootnote{Package}ی پایتونی به کارگزار\LTRfootnote{Server} اصلی اضافه شده است. همچنین برای پیاده‌سازی مدل و انجام محاسبات ریاضی و ماتریسی از کتابخانه‌های نام‌پای\LTRfootnote{NumPy} و سایکیت\LTRfootnote{Scikit-Learn} بهره برده شده است. در بخش‌های بعد به معرفی مختصر هر کدام از این موارد خواهیم پرداخت. لازم به ذکر است که جهت خوانایی بیشتر، از معادل انگلیسی این کتاب‌خانه‌ها استفاده خواهیم کرد.

\subsection{چارچوب \lr{FastAPI}}


\subsection{کتاب‌خانه‌ی \lr{NumPy}}


\subsection{کتاب‌خانه‌ی \lr{Scikit-Learn}} 