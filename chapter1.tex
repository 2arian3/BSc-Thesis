\chapter{مقدمه}

\section{مقدمه}
در صنعت، نگهداری و تعمیرات\LTRfootnote{Maintenance} به تمام فعالیت‌هایی اطلاق می‌شود که بر روی ابزارهای صنعتی انجام می‌شود تا بهره‌وری و عمر این ابزارها افزایش یابد. در سال‌های اخیر، رویکردهای مختلفی برای انجام نگهداری مورد استفاده قرار گرفته است. روش‌های نگهداری زیر، از میان همه‌ی این رویکردها، بیشترین فراوانی استفاده در صنعت را دارند\cite{zhao2022review}:

\begin{itemize}

\item \textbf{نگهداری و تعمیرات اصلاحی\LTRfootnote{Corrective Maintenance}}: به جایگزینی قطعه خراب شده در سیستم می‌پردازد. در این رویکرد، تا زمانیکه فرایند جایگزینی قطعه معیوب به اتمام نرسد، سیستم غیرقابل بهره‌برداری است و تعمیر قطعات بعد از خرابی هزینه‌های قابل توجهی برای صاحبان صنعت به همراه دارد\cite{ran2019survey}.

\item \textbf{نگهداری و تعمیرات جلو‌گیرانه\LTRfootnote{Preventive Maintenance}}: سعی در پیش‌گیری از اتلاف زمان ناشی از توقف اضطراری دارد، اما در عوض ممکن است تعدادی از قطعاتی که هنوز عمر مفید دارند، دور ریخته شوند و اصراف در هزینه و قطعات مصرفی صورت گیرد\cite{ran2019survey}.

\item \textbf{نگهداری و تعمیرات پیش‌بینانه\LTRfootnote{Predictive Maintenance}}: سعی می‌کند مشکلات دو نوع نگهداری و تعمیرات مذکور را حل کند. با استفاده از این روش، زمان عملیاتی هر قسمت دستگاه تخمین زده می‌شود و قطعاتی که توسط سیستم مشکوک به خرابی در آینده هستند تعویض می‌گردند و بنابراین ابزارهای موجود در سیستم به صورت بهینه مورد استفاده قرار می‌گیرند و هزینه‌های تعمیرات بشدت کاهش می‌یابد\cite{zonta2020predictive, ran2019survey}.

\end{itemize}

بدلیل اینکه در نگهداری پیش‌بینانه قطعات در حال خرابی، پیش از وقوع خرابی شناسایی می‌شوند و ناکارآمدی آن بخش به کل سیستم آسیب نمی‌رساند، همانطور که در \cref{fig:maintenance_comparison} مشخص است، با استفاده از این نوع نگهداری، می‌توان مجموع هزینه‌های نگهداری و تعمیرات را به حداقل میزان ممکن رساند\cite{zonta2020predictive}.

\begin{figure}[!h]
\centerline{\includegraphics[width=.8\textwidth]{maintenance_comparison.png}}
\caption{مقایسه‌ی هزینه‌های انواع نگهد‌اری‌ها}
\label{fig:maintenance_comparison}
\end{figure}


\section{تعریف مسئله}
هدف از انجام این پروژه، پیاده‌سازی سیستمی برای اجرا کردن نگهداری پیش‌بینانه بر روی گره‌های موجود در یک اینترنت اشیاء\LTRfootnote{Internet of Things} به هم پیوسته است. رویکردهای مختلفی برای این منظور تا کنون توسط محققان ابداع و مورد استفاده قرار گرفته شده است. از جمله‌ی این موارد می‌توان به تحلیل لرزش\LTRfootnote{Vibration Analysis} اشاره کرد. برای پیاده‌سازی این سیستم همانطور که در \cref{fig:workflow} به تصویر آمده ‌است، نیازمند آنیم که داده‌های لرزش مربوط به گره‌ها را که توسط یک سیستم قابل اتکا\LTRfootnote{Reliable} جمع‌آوری شده است، دریافت کرده و با جدا کردن داده‌های پرت\LTRfootnote{Outlier Data}، از بین بردن تاثیر اختلال\LTRfootnote{Noise} ایجاد شده توسط گرانش و خرابی یا درست‌ کار نکردن حسگر\LTRfootnote{Sensor} اندازه‌گیری لرزش، استخراج ویژگی‌\LTRfootnote{Feature Extraction}های مناسب برای انجام تحلیل روی داده و درنهایت پیشنهاد دادن مدلی برای نحوه‌ی یادگیری ماشین\LTRfootnote{Machine Learning} و تحلیل و مقایسه‌ی داده‌های بدست‌آمده با داده‌های برچسب‌دار\LTRfootnote{Labeled Data}، عمر باقی‌مانده\LTRfootnote{Remaining Useful Lifetime}‌ی دستگاه‌های مختلف را پیش‌بینی کنیم و بر اساس اعداد بدست‌آمده، اقدامات مناسب را برای انجام مراقبت‌های دوره‌ای انجام دهیم و از تحمیل‌شدن هزینه‌های جانبی در آینده جلوگیری کنیم\cite{jung2017vibration}. برای راحتی استفاده از سیستم طراحی‌شده، مستقر ساختن\LTRfootnote{Deploy} سرویس توسعه‌یافته‌شده روی ابر\LTRfootnote{Cloud} و همچنین احراز هویّت مدیر\LTRfootnote{Admin}ان و دروازه\LTRfootnote{Gateway}‌های ارسال‌کننده داده‌ی لرزش کارگزار\LTRfootnote{Server}ی نیز پیاده خواهد شد.

\begin{figure}[!h]
\centerline{\includegraphics[width=1\textwidth]{workflow.png}}
\caption{نمودار جریان کار}
\label{fig:workflow}
\end{figure}


\section{کارهای مشابه}
نگهداری و تعمیرات پیش‌بینانه نسبتاً موضوع نو ظهوری است و عمر کمتری را نسبت به انواع دیگر نگهداری‌های موجود دارد. اما با این‌حال تا به امروز تلاش‌های قابل توجهی برای بکارگیری این نوع از نگهداری در سطح دنیا صورت گرفته است که در اینجا به مواردی که رویکردهای جالبی داشته‌اند اشاره خواهیم کرد. در \cite{tinga2010application} مدلی براي خرابی قطعات مبتنی بر ميزان استفاده از تجهيزات و بار داخلی آنها ارائه شده‌است اما مدل ارائه‌شده محدود به يک مدل تجهيزات است و براي استفاده در ساير مدلها نظارت مجدد لازم است. در \cite{wu2007neural} روندی مبتنی بر شبكه‌های عصبی جهت پيش‌بينی عمر مفيد باقی‌مانده تجهيزات چرخشی ارائه شده‌است اما تنها مختص به اين دسته از تجهيزات است. در \cite{kaiser2009predictive} رويكردی مبتنی بر شبكه عصبی جهت پيش‌بينی زمان نگهداری و تعميرات تجهيزات بر اساس مدل خرابی و كارايی آنها ارائه می‌شود اما در یک محيط شبيه‌سازی‌شده و كنترل‌شده آزمايش شده‌است و در هنگام استفاده در محيط واقعی غيرعملی است.

 بطور كلی در پژوهش‌های يادشده روندهای در پیش‌گرفته‌شده برای پیش‌بینی خرابی و زمان آن، مختص نوع خاصی از تجهيزات است و در يک محيط آزمايشگاهی و كنترل‌شده ارزيابی شده‌اند. در حالی‌كه در اين پروژه با استفاده از داده‌های جمع‌آوری‌شده از قطعات مختلف سعی كرده‌ايم رويكردی كلی و مناسب محيط واقعی و صنعتی ارائه دهيم. همچنین شایان ذکر است که در هیچ‌کدام از پروژه‌های یاد‌شده، سیستم طراحی‌شده روی ابر مستقر نشده‌اند و سرویس‌هایی همانند احراز هویت و برنامه‌ی تحت وب برای آنها طراحی نشده است. این در حالی است که در این پروژه قصد بر این بوده که سیستمی کلی برای مدیریت بهتر قطعات با جلوه‌ای مناسب طراحی گردد و توسعه یابد.
