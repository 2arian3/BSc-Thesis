\chapter{چالش‌ها و محدودیت‌ها}
در این فصل کوتاه، ابتدا چالش‌هایی که در مسیر توسعه‌ی سیستم نگهداری پیش‌بینانه به آنها برخوردیم را عنوان می‌کنیم. در قسمت بعد، محدودیت‌هایی که خواه‌‌نا‌خواه بر روند تکمیل پروژه تاثیراتی گذاشتند را بیان می‌کنیم.

\section{چالش‌ها}
در مراحل مختلف این پروژه، چالش‌هایی سر راه ما قرار گرفتند که برای رفع هر کدام تدابیری اندیشیدیم. در این قسمت به تعدادی از این موارد اشاره می‌کنیم و روش حل هر کدام را توضیح می‌دهیم. 

\begin{itemize}

\item اوّلین مورد انتخاب روش مناسب استقرار سیستم یادگیری ماشین بود. همانطور که در فصل سوم توضیح دادیم، روش‌های مختلفی برای استقرار مدل‌های یادگیری ماشین وجود دارند که بنا به انتخاب هر کدام، نحوه‌ی شروع یادگیری، نحوه‌ی ذخیره‌ و ارسال نتایج پیش‌بینی و روش بروزرسانی مدل یادگیری ماشین متفاوت است. این مورد باید حتما پیش از شروع توسعه‌ی مدل یادگیری ماشین مشخص می‌شد تا برای مواردی که بیان شد، بهترین روش را برگزینیم.

\item نحوه‌ی هموار کردن و استفاده از پارامترهای مناسب پنجره‌ی هان نیز یکی از موارد چالش‌برانگیز دیگر بود. این پارامترها باید با توجه به تعداد نمونه‌های موجود در یک اندازه‌گیری طوری  انتخاب شود که سیگنال خیلی هموار نشود به نحوی که ویژگی ذاتی خود را از دست بدهد. با انتخاب حالت‌های مختلف، مناسب‌ترین گزینه انتخاب شد.

\item پیاده‌سازی الگوریتم مقایسه‌ی بین ویژگی‌ها یکی از مسائل دشوار بود. زیرا باید هنگام پیاده‌سازی به ماهیت سیگنال ویژگی \lr{PSD} و همچنین آهنگ لرزش دستگا‌ه‌ها توجه می‌شد و بر این اساس، الگوریتمی پیشنهاد می‌شد تا به درست‌ترین نحو ممکن اختلاف بین ویژگی‌ها را شناسایی کند. 

\end{itemize}


\section{محدودیت‌ها}
یکی از بزرگ‌ترین محدودیت‌ها در طی انجام این پروژه نبود مجموعه‌ی داده‌‌ی مناسب و برچسب‌دار برای استفاده‌ی مدل یادگیری ماشین بود. از آنجا که مسئله‌ی پیش بینی عمر مفید باقی‌مانده‌ی هر دستگاه، هم یک مسئله‌ی طبقه‌بندی (باید کلاس کاری دستگاه مشخص شود) و هم یک مسئله‌ی رگرسیون (عمر باقی‌مانده‌ی دستگاه باید مشخص شود) می‌باشد، نیازمند دو مجموعه داده‌ی کامل برچسب‌دار بودیم که فراهم نشد. یکی از مهم‌ترین دلایلی که مجموعه داده‌ی آماده برای این نوع مسئله در دسترس عموم قرار ندارد این است که اکثر پروژه‌هایی که در این باب انجام شده‌اند صنعتی بوده و رویکرد هر کدام و نحوه‌ی طبقه‌بندی و پارامترهای دخیل (لرزش، دما، رطوبت، صدا، ارتفاع از سطح دریا) در تحلیل داده برای هر کدام متفاوت است.